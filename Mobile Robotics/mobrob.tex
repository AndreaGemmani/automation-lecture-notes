\documentclass{book}
\usepackage{amsmath, amsthm, graphicx, amsfonts, float}
\usepackage[english]{babel}
\graphicspath{ {./images/} }

\usepackage{geometry}
 \geometry{
 a4paper,
 total={170mm,237mm},
 left=20mm,
 top=30mm,
 }
 \usepackage[hidelinks]{hyperref}

\newcommand\at[2]{\left.#1\right|_{#2}}
\DeclareMathOperator{\sgn}{sgn}
\DeclareMathOperator{\col}{col}
\DeclareMathOperator{\des}{des}
\DeclareMathOperator*{\argmax}{arg\,max}
\DeclareMathOperator*{\argmin}{arg\,min}
\DeclareMathOperator{\im}{Im}
\newcommand{\notimplies}{%
  \mathrel{{\ooalign{\hidewidth$\not\phantom{=}$\hidewidth\cr$\implies$}}}}
\newcommand{\R}{\mathbb{R}}
\newcommand{\N}{\mathbb{N}}
\newcommand{\deriv}[1]{\displaystyle\frac{d}{d #1}}
\newcommand{\traj}{(\bar{\mathbf{x}},\bar{\mathbf{u}})}


\theoremstyle{definition}
\newtheorem{definition}{Definition}[section]
\newtheorem{theorem}{Theorem}[section]
\newtheorem{proposition}{Proposition}[section]
\theoremstyle{remark}
\newtheorem*{remark}{Remark}
\theoremstyle{remark}
\newtheorem*{notation}{Notation}

\title{Industrial Robotics M}
\author{Dante Piotto}
\date{spring semester 2023}


\begin{document}
\chapter{Mobile Robot Control}
\section{Configuration Space}
\begin{itemize}
    \item It has dimensions equal to the number of parameters needed to uniquely describe the configutation of a mobile robot 
    \item Heavily dependent on the structure of the considered robot 
    \item Equivalent to the joint space for manipulators
\end{itemize}

\section{Constraints}
\begin{definition}[Constraint]
    A constraint is any condition imposed to a material system that prevents it from assuming a generic position and/or act of motion
\end{definition}
\begin{definition}[holonomic constraints]
    A material system is subject to \emph{holonomic constraints} if finite relations between the coordinates of the system are present (position costraints) or if differentiable/integrable relations between the coordinates of the system are present
\end{definition}
\begin{definition}[non holonomic constraints]
    A constraint is said to be \emph{non-holonomic} if the differential relation between the coordinates is not reducible to finite form.
\end{definition}
Some insight on non-holonomic constraints: 
\begin{itemize}
    \item They cannot be fully integrated 
    \item They cannot be written in the configuration space 
    \item They do not restrict the space of configurations but the instant robot mobility
\end{itemize}

\subsection{Constraints in Pfaffian form}
\begin{itemize}
    \item Constraint vector equation: $a(q)\dot{q}=0$ (1 wheel)
    \item Constraint matrix equation $A(q)\dot{q}=0$ ($N$ wheels)
\end{itemize}
\begin{definition}[Pfaffian constraint]
    A constraint that can be written in \emph{Pfaffian Form} (i.e. $A(q)\dot{q}=0$), is called a \empf{Pfaffian Constraint}
\end{definition}
\begin{remark}
    Admissible speeds may be generated by a matrix $G(q)$ such that: 
    \[
        \im(G(q))=\ker(A(q)),\forall q \in \mathbb{C}
    \]
    where $\mathbb{C}$ is the configuration space
\end{remark}

\subsection{Types of vehicles}
\begin{definition}[Unicycle model]
    A unicycle is a vehicle with a single adjustable wheel.
\end{definition}
\begin{tabular}{ c c c }
    \toprule
    Configuration & Constraints & Pfaffian form \\
    \midrule
    $q = \begin{bmatrix}
        x \\ y \\ \theta
    \end{bmatrix}$ & $\dot{x}\sin\theta - \dot{y}\cos\theta=0$ & $
        A(q)=\begin{bmatrix}
            \sin\theta & -\cos\theta & 0
    \end{bmatrix} $ \\
    \bottomrule
\end{tabular}
\begin{definition}[Bycicle kinematic model]
    A bycicle is a vehicle haveing a castor and a fixed wheel with their rotation axes perpendicular to the longitudinal plane
\end{definition}\begin{definition}[Bycicle kinematic model]
    A bycicle is a vehicle haveing a castor and a fixed wheel with their rotation axes perpendicular to the longitudinal plane
\end{definition}
\begin{tabular}{ c c c }
    \toprule
    Configuration & Constraints & Pfaffian form \\
    \midrule
    % Configuration
    $q = \begin{bmatrix} 
        x \\ y \\ \theta \\ \gamma
    \end{bmatrix}$ & 
    % Constraints
    $\left\{\begin{aligned}
        &\dot{x}_f\sin(\theta+\gamma)-\dot{y}_f\cos(\theta+\gamma)  = 0\\
        &\dot{x}_r\sin(\theta)-\dot{y}_r\cos(\theta) =0
    \end{aligned}\right.$ & 
    %Pfaffian form
    $
        A(q)=\begin{bmatrix}
            \sin\theta & -\cos\theta & 0 & 0\\
            \sin(\theta+\gamma) & -\cos(\theta+\gamma) & -L\cos\gamma & 0
    \end{bmatrix} $ \\
    \bottomrule
\end{tabular}

\section{Motion Control}
\subsection{Trajectory Following}
\subsubsection{Input-Output State Feedback Linearization}
We define a point $B$ outside the wheel's axle so that we can control the point $b$ which \emph{pulls} the vehicle
\[
    \left\{\begin{aligned}
        &x_b = x_r + b\cos\theta_r \\
        &y_b = y_r + bsin\theta_r
    \end{aligned}\right.\qquad  b\neq 0
\]
The point $B$ is not constrained in any way, therefore it can instantly move in any direction. We can define two inputs $(v_{x,b}, v_{y,b})$ to control the system: 
\[
    \left\{\begin{aligned}
        &\dot{x}_b = v_{x,b} \\
        &\dot{y}_b = v_{y,b}
    \end{aligned}\right.\qquad  b\neq 0
\]
which are related to the unicycle's configuration as follows: 
\[
    \left\{\begin{align}
            &\dot{x}_b = \dot{x}_r-b\omega\sin\theta_r = v\cos\theta_r-b\omega\sin\theta_r \\
            &\dot{y}_b = \dot{y}_r-b\omega\cos\theta_r = v\cos\theta_r-b\omega\cos\theta_r \\
    \end{align}\right. \qquad b\neq 0
\]

























\end{document}

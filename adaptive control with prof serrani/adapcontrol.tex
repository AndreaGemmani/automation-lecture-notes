\documentclass{book}
\usepackage{amsmath, amsthm, graphicx, amssymb, mathtools, float}
\usepackage[english]{babel}
\DeclareMathOperator*{\argmax}{arg\,max}
\DeclareMathOperator*{\argmin}{arg\,min}

\usepackage{geometry}
 \geometry{
 a4paper,
 total={170mm,237mm},
 left=20mm,
 top=30mm,
 }

\title{Learnig and Estimation of Dynamical Systems}
\author{Dante Piotto}
\date{Spring semester 2023}

\begin{document}
We will treat Model-based control. 

\begin{itemize}
    \item Ideas 
    \item examples
    \item mrac (model reference adaptive control) 
    \item stability of adaptive control systems
\end{itemize}

Let us call $\mathcal{P}$ the plant model. We define a number of signals as input to the model (control input $u$, disturbance $d\in \mathbb{R}^q$, reference signals $r\in\mathbb{R}^s$) and as output (measured output $y\in\mathbb{R}^p$, error to the reference $e\in\mathbb{R}^r$). The goal is to determine a feedback control $\mathcal{C}$ in a way st certain criteria are met for the system. We start by restricting the class of signals $d$ and $r$ 
\begin{itemize}
    \item $r,d \in \mathcal{L}_{\infty}$
\end{itemize}
Desired performance: 
\begin{itemize}
    \item $e(t),t\geq0$ bounded ("small" if $d,r$ are "small")
    \item $d(t)=0 \implies \lim_{t\to\infty}\|e(t)\|=0$
\end{itemize}

For adaptive control we assume to have a parameterized family of plant models $\mathcal{P}(\theta)$ where $\theta$ is a set of parameters that range within bounded sets, in order to deal with the issue of parametric uncertainties. One option to deal with having infinitely many plants given by the variation of the parameters is to have infinitely many parameterized controllers that are tuned to solve the control problem.

We state 2 problems: 
\begin{enumerate}
    \item Given $r(\cdot)\in\mathcal{L}_\infty$ design a controller $\mathcal{C}(k*)$ with $k^*$ fixed values such that when $d(\cdot)=0$ the closed loop system is such that \begin{enumerate}
            \item all "internal" trajectories are bounded 
            \item $\lim_{t\to\infty}\|e(t)\|=0$
        \end{enumerate} 
    \item when $d(\cdot)\neq 0,d(\cdot)\in \mathcal{L}_\infty$ \begin{enumerate}
            \item All internal trajectories are bounded 
            \item $e(t)$ is ultimately bounded, meaning 
                \[
                    \forall \varepsilon>0 \exists \delta>0 :\forall d(\cdot)\in\mathcal{L}:\|d(\cdot)\|\leq\delta \text{then} limsup_{t\to\infty}\|e(t)\|\leq \varepsilon
                \]
        \end{enumerate}
\end{enumerate}

\section{Model Reference Adaptive control}
we start with a system called the reference model $\mathcal{M}_r$
which has as input the reference signal $r$ and as output $y_r(t)$, while the actual ssytem $\mathcal{P}(\theta)$ takes as input the control input $u$ and as output the measured output $y(t)$. We define the tracking error as 
\[
    e=y-y_r
\]
The goal for the control system $\mathcal{C}(k)$ is 
\begin{enumerate}
    \item All internal signlas bounded if $r(t)\in \mathcal{L}_{infty}, y_r(t)\in \mathcal{L}_{infty}$
    \item $\lim_{t\to\infty}\|e(t)\|=0$
\end{enumerate}
$\mathcal{M}_r$ is typically a stable filter, and is called the "handling quality" in aerospace. The reference model defines the desired behaviour of the system. 
$\mathcal{C}(k)$ must have a structure such that the control problem is solvable. We introduce the 
\subsubsection{Certainty equivalence princlple}
You need to find matching conditions between the gains of the controller and the parameter vector of the plant, meaning a function between $k$ and $\theta$ : 
\[
    k^*=k^*(\theta) \\
    k^*:\mathbb{R}^p \to \mathbb{R}^\mu
\]
if $k=k^*(\theta)$ then $\mathcal{C}(k)$ solves the mrac problem for $\mathcal{P}(\theta)$
A tuning mechanisim $\tau$ (\emph{update law}) is used to fix $k$ in real time $k\to\hat{k}(t)$: 
\[
    \dot{\hat{k}}(t)=\tau
\]
\begin{enumerate}
    \item start with a matching condition $k^*=k^*(\theta)$ (the problem will be solved if i know $\theta$) 
    \item replace $k$ with a time varying $\hat{k}(t)$ 
    \item find $\tau$ in $\dot{\hat{k}}(t)=\tau(\dots)$ such that $\hat{k}(t)\to k^*$ as $t\to\infty$
    \item use $\mathcal{C}(\hat{k})$ to control the system
\end{enumerate}
Attention must be paid to the fact that the controller becomes a dynamical system and its transient may destroy the stabilty of the system. 

Two ways of tackling the problem are available 
\begin{itemize}
    \item Direct adaptive control: we find the matching condition and design $\tau: \hat{k}(t)\to k^*$ (directly updating the controller gains)
    \item Indirect adaptive control: estimate the parameters of the plant $\hat{\theta}$ online and obtain the estimate of the gain from the matching condition $\hat{k}(t)=k^*(\hat{\theta})$
\end{itemize}

\subsection{Example: SISO scalar plant model}
\[
    G(s)=\displaystyle\frac{b}{s+a} \qquad \theta=\begin{pmatrix}
        a \\ b
    \end{pmatrix}
\]
Assumption: sign of $b$ is known ($b>0$) and $\exists b_0>0:b\geq b_0$
\[
    G_r(s)=\displaystyle\frac{\beta_m}{s+\alpha_m}\quad \beta_m>0,\alpha_m>0
\]
\subsubsection{Matching conditions }
\begin{gather*}
    \dot{y}_r=-\alpha_my_r+\beta_mr\\
    \dot{y}=-ay+bu\\
    e=y-y_r
\end{gather*}
Change of coordinates: 
\[
    \dot{e}=-ay+bu+\alpha_m y_r- \beta_mr
\]
substituting $y_r=y-e$ 
\[
    \dot{e}=-ay+bu\alpha_my-\alpha_me-\beta_mr
\]
cleaning up the expression 
\[
    \dot{e}=-\alpha_me+(\alpha_m-a)y+bu-\beta_mr
\]
If only the first term on the right was present we wuold have an exponentially descending system. The control can therefore be chosen as 
\[
    u=k_1y+k_2r
\]
the first term being a fb and the second being ff. We obtain 
\[
    \dot{e}=-\alpha_me+(\alpha_m-a+bk_1)y+(bk_2-\beta_m)r
\]
The matching conditions are therefore 
\[
    k_1^* = \displaystyle\frac{a-\alpha_m}{b} \qquad k_2^* = \displaystyle\frac{\beta_m}{b}
\]

Because $a$ and $b$ are not known, we must use estimates for the control:
\[
u=\hat{k}_1(t)y + \hat{k}_2(t)r 
\]
leading to
\begin{gather*}
    \dot{e}=-\alpha_me+(\alpha_m-a+b\hat{k}_1)y+(b\hat{k}_2-\beta_m)r=\\
    =-\alpha_me+(\alpha_m-a+b\hat{k}_1+bk_1^*-bk_1^*)y+(b\hat{k}_2-\beta_m+bk_2^*-bk_2^*)r=\\
    =-\alpha_me+b(\hat{k}_1-k_1^*)y+b(\hat{k}_2-k_2^*)r = \\
    =-\alpha_me+b\tilde{k}_1y + b\tilde{k}_2r
\end{gather*}
\subsubsection{desing of $\tau$}
We must find $\tau_1,\tau_2$ s.t. for $\dot{\hat{k}}_1=\tau_1,\dot{\hat{k}}_2=\tau_2$, $\tilde{k}_1,\tilde{k}_1$ converge. Directly assigning the dynamics is impossible as knowledge of $k_1^*,k_2^*$ is necessary. We rewrite the error derivative as 
\[
    \dot{e}=-\alpha_me+b\Phi^T(y,r)\tilde{k} \qquad \Phi^T(y,r)=(y\quad r) \qquad \tilde{k}=(\tilde{k}_1 \quad \tilde{k}_2)^T
\]
The second term on the right is a regressor. The system is non linear, as any adaptive system. 

\subsubsection{Stability analysis}
We use a Lyapunov approach as the system is non-linear. 
\begin{gather*}
    \dot{e}=-\alpha_me+b\Phi^T(e+y_r,r)\tilde{k} \\
    \dot{\tilde{k}}=\tau
\end{gather*}
Goal: find $\tau$ so that the equilibrium $(0,0)$ of the system is asymptotically stable. If we can make the system so, we obtain the goals of the control + $\tilde{k}(t)\to 0$. 
\\Lyapunov candidate function
\[
    V(e,\tilde{k})=\displaystyle\frac{1}{2}e^2+\displaystyle\frac{b}{2\gamma}\tilde{k}^T\tilde{k} \qquad with \gamma>0
\]
The derivative is 
\[
    \dot{V}(e,\tilde{k})=-\alpha_me^2+eb\Phi^T(e+y_r,r)\tilde{k} + \displaystyle\frac{b}{\gamma}\tilde{k}^Tr
\]
The first term is negative, the second is bad, the third is so-so
% Lyapunov function is just a glorified energy
\[
    \dot{V}(e,\tilde{k})=-\alpha_me^2+\tilde{k}^Tb\displaystyle\frac{1}{\gamma}[\gamma\Phi(e+y_r,r)e+\tau]
\]
$\gamma$ and the regressor are knwon, so we define 
\[
    \tau=\gamma\Phi(e+y_r,r)e
\]
leading to 
\[
    \dot{V}(e,\tilde{k})=-\alpha_me^2
\]
which is not negative definite as it is identically 0 for $e=0$ and any value of $\tilde{k}$, it is therefore negative semi-definite. However the system $(e,\tilde{k})=(0,\vec{0})$ is stable in the Lyapunov sense.
This means that, take arbitrarirly $e_0,\tilde{k}_0$ initial conditions, the derivative of the lyapunov function of the corresponding solution $e(t),\tilde{k}(t)$ is negative semi-definite, meaning the energy of the system is non increasing, meaning $V(e(t),\tilde{k}(t))\leq V(e_0,\tilde{k}_0)=M_0$, giving us global boundedness of the system. 

What happens to $e(t) and \tilde{k}(t)$ as $t\to\infty$? Wec cannot apply the La Salle criterion as the system is time variant(due to $r$ being an external time varying signal) and the La Salle criterion only applies to time invariant or periodic systems. We use the La Salle-Yoshizawa criterion 
\subsubsection{La Salle-Yoshizawa theorem}
An extension of the La Salle criterion that applies to time variant systems. \\
\begin{gather*}
    \dot{x}=f(t,x) \qquad x(0)=x_0 \qquad f:\mathbb{R}\times \mathbb{R}^n \to\mathbb{R}^n \quad f(\cdot,x) \text{is piecewise continuous } \forall x \in \mathbb{R}^n \\
    f(t,\cdot) \forall t \in \mathbb{R} \text{ is locally Lipschitz in the variant argument}
\end{gather*}
We assume that the origin is an equilibrium ($x=\vec{0}$)\\
We start with a lyapunov function candidate $V:\mathbb{R}\times\mathbb{R}^n\to \mathbb{R}^n$ with the following properties: 
\begin{itemize}
    \item $V\in\mathcal{C}^1$ (continuously differentiable)
    \item positive definite 
    \item radially unbounded 
    \item decrescent
\end{itemize}
Let us start with a Lyapunov function 
\[
    V(t,x)=\displaystyle\frac{1}{2}e^2+\displaystyle\frac{b}{2\gamma}\tilde{k}^T\tilde{k} \quad
\]
this respects the assumptions made so far 
\\Assumption 2: 
\[
    \dot{V}(t,x)=\displaystyle\frac{\partial V}{\partial t}+\displaystyle\frac{\partial V }{\partial x}f(t,x) \leq -w_3(x) \quad w_3(x)\geq 0 \forall x\in\mathbb{R}^n
\]
meaning it is negative semi-definite.\\
The following is the result of the thm: 
\begin{enumerate}
    \item The equilibrium $x=0$ is globally stable 
        \[
            \forall R>0 \exists M_R:\forall x_0\in \mathcal{B}_R(0) text{the solution } x(t,x_0) |x(t,x_0)|\leq M_R \forall t \geq0
        \]
        \item \[
                \forall x_0\in\mathbb{R}^n \quad \lim_{t\to\infty}w_3(x(t,x_0))=0
        \]
\end{enumerate}
By La Salle-Yoshizawa $\alpha_me^2(t)\to 0 $ as $t\to\infty$ so $\lim_{t\to\infty}|e(t)|=0$
\\We have achieved convergence of the error to 0 and boundedness of the state (verified for $\tilde{k}$ through Lyapunov analysis and for the other states because the system is linear)

\subsubsection{Theorem of total stability}
Suppose you have a NL time-varying system $\dot{x}=f(t,x)$ with the same assumptions on $f$ as La Salle-Yoshizawa. Now assume that \[
    f(t,\vec{0})=\vec{0} \forall t\in\mathbb{R}
\]
and assume that the origin $x=\vec{0}$ is a uniformly locally asymptotically stable equilibrium. Then, for the perturbed system 
\[
    \dot{x}=f(t,x)+g(t,x) \quad x(t_0)=x_0
\]
the following holds: 
\begin{gather*}
    \forall \varepsilon>0 \quad \exists \delta_1(\varepsilon), \delta_2(\varepsilon)>0 \qquad \text{st} \forall x_0\in\mathcal{B}_{\delta_1}=\{x_0:|x_0|<\delta_1\}\\
    \forall g(t,x): |g(t,x)|<\delta_2(\varepsilon), \forall \geq 0, \forall x\in\mathcal{B}_\varepsilon then |x(t,x_0)|\leq \varepsilon \quad \forall t \geq 0
\end{gather*}
% poor people's robustness
So far we have proven that $e=0$ is attractive, however we do not know aobut $\tilde{k}=0$. In general, it is not attractive. 

\subsection{How robust is adaptive control?}
\begin{gather*}
    \dot{e}=-\alpha_me+b\Phi^T(e+y_r,r)\tilde{k}\\
    \dot{\tilde{k}}=-\gamma\Phi(e+y_r,r)e
\end{gather*}
We know that the equilibrium $(e,\tilde{k})=(0,\vec{0})$ is uniformly stable. Is it attractive? $e(t)\to 0$ as $t\to\infty$. What about $\tilde{k}(t)$? It depends. 
\[
    \begin{pmatrix}
        \dot{e} \\ \dot{\tilde{k}}
    \end{pmatrix} = \begin{pmatrix}
    -\alpha_m & b\Phi^T(\cdots) \\ -\gamma\Phi(\cdots) & 0
    \end{pmatrix} \begin{pmatrix}
    e \\ \tilde{k}
    \end{pmatrix}
\]
which is a skew symmetric matrix, which is always related to conservation of energy in physics. 
\\If the regressor is persistently exciting and La Salle is applicable we obtain attracivity of the origin.
















\end{document}
